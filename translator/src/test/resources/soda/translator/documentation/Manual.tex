\documentclass[12pt,a4paper]{article}

\usepackage[utf8]{inputenc}
\usepackage{xcolor}
\usepackage{hyperref}
\usepackage{listings}

\lstdefinelanguage{Soda}{
    morekeywords={lambda, any, def, if, then, else, match, case, class, extends, abstract, end, this, subtype, supertype, false, true, not, and, or, package, import, theorem, directive, @tailrec, @override, @new},
    sensitive=true,
    morecomment=[s]{/*}{*/},
   morestring=[b]"
}

\lstset{frame=tb,
    language=Soda,
    aboveskip=3mm,
    belowskip=3mm,
    showstringspaces=false,
    columns=flexible,
    basicstyle={\small\ttfamily},
    numbers=none,
    numberstyle=\tiny\color{gray},
    keywordstyle=\color{blue},
    commentstyle=\color{gray},
    stringstyle=\color{teal},
    breaklines=true,
    breakatwhitespace=true,
    tabsize=3
}

\begin{document}

\begin{lstlisting}
package soda.manual


\end{lstlisting}

This is a Soda tutorial written in Soda.
Copyright 2020--2024 Julian Alfredo Mendez
 Version: 2024-02-01


\begin{lstlisting}



\end{lstlisting}

This tutorial is itself a "Hello world!" program.The piece of code that prints the message is at the end of the file. 


\begin{lstlisting}



\end{lstlisting}

Source code written in Soda is intended to be descriptive and readable.It is usually written in different files, and each file has `blocks`.
These blocks are pieces of code that have some meaning.
A block should be short, maybe less than 10 lines.
However, it is more important to make things clear than concise.
These are examples of blocks:
1. a constant or function definition
2. the beginning of a class definition
3. the end of a class definition
4. a block declaration of abstract constants and functions
5. a block of imports
6. a package declaration
8. a comment 7. a class alias


\begin{lstlisting}



\end{lstlisting}

To declare a class, just add `class` before a class name, and end it with `end`. It is a goodpractice to include `abstract` in the class, to explicitly state whether constants and
functions are required to instantiate a class. In the example below, no constants or
functions ar required.
The name could be a noun or an adjective, but it should not be a verb. For the class name, it is recommended to use camel case style starting with a capital letter.


\begin{lstlisting}


class Shape

  abstract

end


\end{lstlisting}

The reserved word `class` declares a type, a namespace, and a default constructor.The default constructor is the name of the class with an underscore as suffix.
For example, the constructor name for `Shape` is `Shape_`.
In addition, it is possible to use a function `mk` to instantiate a class. The notation looks
parentheses. as a static function of the type. For example,  for `Shape` is `Shape .mk` without


\begin{lstlisting}


class Movable

  abstract

end


\end{lstlisting}

It is recommended to indent the constants and functions declared inside. 


\begin{lstlisting}


class EqualsExample


\end{lstlisting}

Note that `abstract` does not contain any constants or functions in this example, and thedeclared constants and functions are in a different block. 


\begin{lstlisting}


  abstract


\end{lstlisting}

The constant name should be a noun. A constant does not have parameters and it is declared with the equals sign (`=`).For the constant name, it is recommended to use snake case and start in lowercase.


\begin{lstlisting}


  my_number : Int = 2


\end{lstlisting}

A function has parameters and a type. Functions, even with empty parameters, are evaluatedevery time they are invoked. The standard way of declaring and invoking a function
example, use `f (x) (y)` instead of `f(x, y)`. with multiple parameters is with parameters separated by spaces and not by commas. For


\begin{lstlisting}


  f (x : Int) (y : Int) : Int = 2 * x + y


\end{lstlisting}

Constants are only evaluated once, which is the first time they are needed. 


\begin{lstlisting}


  first_result : Int = f (12) (4)


\end{lstlisting}

In a function call, the parameters can be specified with the colon-equals sign (`:=`).This is especially recommended when several parameters are of the same type. 


\begin{lstlisting}


  second_result : Int = f (x := 20) (y := -10)

end


\end{lstlisting}

A class can extend other classes by using `extends`.Abstract classes cannot be instantiated but can be extended.
Conversely, concrete classes cannot be extended but can be instantiated.
Concrete classes are declared with parentheses `(` and `)`.
A class does not need to define all its constants and functions. It is recommended that concrete classes do not have a body, because this cannot be extended.


\begin{lstlisting}


class RegisteredPerson


\end{lstlisting}

leaving lines between the declared attributes. The block starting with `abstract` denotes a constant or function that needs to be definedin extending classes. Only one `abstract` block should be defined per class, without


\begin{lstlisting}


  abstract
    first_name : String
    last_name : String


\end{lstlisting}

If a constant or function is not meant to be exported, its name should start with anunderscore. 


\begin{lstlisting}


  _separator = " "


\end{lstlisting}

Strings can be concatenated by using the plus sign (`+`). 


\begin{lstlisting}


  full_name = first_name + _separator + last_name

end

class Agent

  abstract
    identifier : String

end


\end{lstlisting}

`class`, which need to be in a different block from `abstract`. A concrete class needs as parameters all the constants and functions that have not beendefined in its super classes. Please, note that `extends` has to be in the same block as


\begin{lstlisting}


class RegisteredPersonAgent
  extends
    Agent
    RegisteredPerson

  abstract
    identifier : String
    first_name : String
    last_name : String

end

class Element


\end{lstlisting}

In this class, `accept` is a function that takes an object of type `Visitor` and returnsan object of type `Boolean`. This is indicated with the type arrow `->`. 


\begin{lstlisting}


  abstract
    accept : Visitor -> Boolean

end

class Visitor


\end{lstlisting}

In this class, `visit` is a function define from `Element` to `Boolean`. 


\begin{lstlisting}


  abstract
    visit : Element -> Boolean

end

class Item
  extends Element

  abstract
    identifier : Int


\end{lstlisting}

It is possible to refer to an object instance by using `this`.The dot (`.`) notation is the standard way of accessing attributes and methods of an
object. The space before the dot is to improve readability, but it is not necessary.
Lambda functions are declared with `lambda` and a long right arrow (`-->`).
   Please, notice the difference between the type arrow (`->`) and the lambda arrow (`-->`).


\begin{lstlisting}


  accept : Visitor -> Boolean =
    lambda visitor -->
      visitor .visit (this)

end

class PersonName

  abstract
    name : String


\end{lstlisting}

diamond-shaped class hierarchy. It is possible to override a function by using the `@override` annotation.This is intended only for exceptional cases, like the `toString` function, or a


\begin{lstlisting}


  @override
  toString = name

end


\end{lstlisting}

A class can be parameterized using square brackets ('[' and ']').The parameter needs to be of type Type. 


\begin{lstlisting}


class MyList [A : Type]

  abstract

end


\end{lstlisting}

It is possible to have multiple type parameters. 


\begin{lstlisting}


class MyPair [A : Type] [B : Type]

  abstract
    fst : A
    snd : A

end


\end{lstlisting}

For example, `A subtype B` means that `A` is a subtype of `B`. The parameter type can be constrained using `subtype` and `supertype`.In that case, it is not necessary to declare the parameter to be of type Type.


\begin{lstlisting}


class ShapePainter [A subtype Shape]

  abstract

end


\end{lstlisting}

 This contains the examples shown in the manual.


\begin{lstlisting}


class Manual

  abstract


\end{lstlisting}

The first line in this file is the package declaration. It contains the reserved word`package` followed by the package name. The recommended package naming convention is to
The package declaration usually goes in a separate file called `Package.soda`. start with `soda.`, to avoid name conflicts when it is translated to Scala.


\begin{lstlisting}



\end{lstlisting}

It is possible to import classes by listing them under the `import` reserved word.Imported classes can also be declared in the `Package.soda` file, when they are global
may be used to produce side effects. for the whole package. The list of imported classes can be used to control which classes


\begin{lstlisting}


  import
    java.util.Date

  a = 1

  b : Int = 2


\end{lstlisting}

An instance of a JVM class can be created with the `@new` annotation. If the code istranslated to Scala 3, this annotation is not required. 


\begin{lstlisting}


  now : Date = @new Date ()

  plus_one (x : Int) : Int = x + 1


\end{lstlisting}

A piecewise function can be defined using an `if`-`then`-`else` structure. The condition inthe `if` is evaluated, and then only the corresponding branch is evaluated. 


\begin{lstlisting}


  max (x : Int) (y : Int) : Int =
    if x > y
    then x
    else y


\end{lstlisting}

Scala sequences (`Seq`) can be used, as well as other basic Scala classes. 


\begin{lstlisting}


  plus_one (sequence : Seq [Int] ) : Seq [Int] =
    sequence .map (lambda element --> element + 1)


\end{lstlisting}

A synonym for `lambda` is `any`, which sometimes brings more readability. 


\begin{lstlisting}


  plus_two (sequence : Seq [Int] ) : Seq [Int] =
    sequence .map (any element --> element + 2)


\end{lstlisting}

Boolean values `false` and `true` are available. 


\begin{lstlisting}


  my_not (x : Boolean) : Boolean =
    if x
    then false
    else true

  my_and (x : Boolean) (y : Boolean) : Boolean =
    if x
    then y
    else false

  my_or (x : Boolean) (y : Boolean) : Boolean =
    if x
    then true
    else y


\end{lstlisting}

Boolean values have the standard `not`-`and`-`or` functions. 


\begin{lstlisting}


  my_xor (x : Boolean) (y : Boolean) : Boolean =
    (x or y) and not (x and y)


\end{lstlisting}

It is possible to use pattern matching with `match` and `case`.The result of the matching case is put after a long double arrow `==>`.
The order matters, so the remaining cases are capture by the last variable.
the type arrow (`->`). Please notice the difference between the case arrow (`==>`), the lambda arrow (`-->`), and


\begin{lstlisting}


  if_then_else [A : Type] (condition : Boolean) (if_true : A) (if_false : A) : A =
    match condition
      case true ==> if_true
      case false ==> if_false


\end{lstlisting}

A constant or function name starting with underscore indicates that the constant orfunction is private, and therefore is not visible outside the class. 


\begin{lstlisting}


   _my_private_function (x : Float) : Float =
     x * x + x + 1


\end{lstlisting}

detected and optimized when it is translated to Scala. A tail recursive function cannot be declared inside another function, and its name shouldstart with underscore. The annotation `@tailrec` helps ensuring that the tail recursion is


\begin{lstlisting}


  @tailrec
  _tailrec_sum (n : Int) (accum : Int) : Int =
    if n < 0
    then accum
    else _tailrec_sum (n - 1) (n + accum)

  sum (n : Int) =
    _tailrec_sum (n) (0)

end


\end{lstlisting}

The class `Fold` shows a 'left fold', which is a functional approach to iterations. Startingwith an initial value (`initial`), it traverses a sequence (`sequence`) applying a function
computation. (`next`) using the current element in the sequence and the result of the previous


\begin{lstlisting}


class Fold

  abstract


\end{lstlisting}

Note that the type parameters need to specified in the function call:`_tailrec_foldl [A] [B]`... The sequence constructor `+:` is defined by `Seq`.
`List`. `Nil` is the constructor for an empty `Seq`. This is equivalent to the more common constructor `::`, when it is used for instances of


\begin{lstlisting}


  _tailrec_foldl [A : Type] [B : Type] (sequence : Seq [A] ) (current : B)
      (next : B -> A -> B) : B =
    match sequence
      case Nil ==> current
      case (head) +: (tail) ==>
        _tailrec_foldl [A] [B] (tail) (next (current) (head) ) (next)


\end{lstlisting}

Ideally, each object should have one responsibility or purpose. The function `apply`defines the main responsibility of an object. 


\begin{lstlisting}


  apply [A : Type] [B : Type] (sequence : Seq [A] ) (initial : B) (next : B -> A -> B) : B =
    _tailrec_foldl [A] [B] (sequence) (initial) (next)

end


\end{lstlisting}

translation to Scala. A piece of code of the destination language can be included with the reserved word`directive`. In this example, we can define the concept of successor for integers for the


\begin{lstlisting}


directive scala
object Succ_ {
  def unapply (n : Int) : Option [Int] =
    if (n <= 0) None else Some (n - 1)
}


\end{lstlisting}

The class `Range` generates a sequence of length `n` of consecutive natural numbers startingfrom 0. 


\begin{lstlisting}


class Range

  abstract


\end{lstlisting}

Note that `Range` processes any negative number as 0. This is done by `Succ_`, which onlyreturns values for positive values of `n`. 


\begin{lstlisting}


  _tailrec_range (non_negative_number : Int) (sequence : Seq [Int] ) : Seq [Int] =
    match non_negative_number
      case Succ_ (k) ==>
        _tailrec_range (k) ( (k) +: (sequence) )
      case _otherwise ==> sequence

  apply (length : Int) : Seq [Int] =
    _tailrec_range (length) (Nil)

end


\end{lstlisting}

`Factorial` shows an example of how to compute the function factorial using a left fold. 


\begin{lstlisting}


class Factorial

  abstract

  fold = Fold_ ()

  range = Range_ ()

  apply (n : Int) : Int =
    fold .apply [Int] [Int] (range .apply (n) ) (1) (
      lambda accum -->
        lambda k --> (accum * (k + 1) ) )

end


\end{lstlisting}

The main class has to be named `Main` and requires a `main` function that receives an`Array [String]` and returns a `Unit`. Only one main class per package is allowed. 


\begin{lstlisting}


class Main


\end{lstlisting}

An output to the standard output can be sent with a `println` command.This is a shorter form of JVM's `System.out.println`. 


\begin{lstlisting}


  main (arguments : Array [String] ) : Unit =
    println ("Hello world!")

end


\end{lstlisting}

The main class has an extending concrete class. The class that needs to be invoked in atranslation to Scala is `EntryPoint`. 


\begin{lstlisting}


\end{lstlisting}

\end{document}

