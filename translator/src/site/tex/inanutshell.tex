\section{In A Nutshell}

The idea of \Soda is a language that can help human describe things in a way a computer can understand.
In short, it is a specification with prototyping.

Learning a programming language and implement something can be relatively easy.
The challenge comes in writing code that is correct, and that humans can easily understand.

The key feature of \Soda is that it has be clear for humans.

Some of the most commonly used mathematical operations are addition, subtraction, multiplication, and division.
For example, a formula like
\begin{lstlisting}[label={lst:exampleAddition}]
  1 + 2 + 4 + 8
\end{lstlisting}
is universally understood, without needing an explanation of what '+' means.

It could be useful to give a name to a such quantity, defining a \emph{constant}.
For example,
\begin{lstlisting}[label={lst:exampleBindingToConstant}]
  a = 1 + 2 + 4 + 8
\end{lstlisting}

And it is also good using other constants in the definition:
\begin{lstlisting}[label={lst:exampleBindingToMultipleConstants}]
  a = 1 + 2 + 4 + 8
  b = a + 16
\end{lstlisting}

The order does not need to be relevant, as long as they are in the same \emph{block}.
Thus, the previous piece is the same as:
\begin{lstlisting}[label={lst:exampleBindingToMultipleConstantsDifferentOder}]
  b = a + 16
  a = 1 + 2 + 4 + 8
\end{lstlisting}

Another useful thing is defining functions.
In principle, there are two direct ways of defining a function defined on the integers.
Something like:
\begin{center}
    $f: Int \to Int$ \\
    $f(x) = x + 16$
\end{center}

However, if we want to condense both the domain definition and the function itself, then we can write:
\begin{lstlisting}[label={lst:exampleFunctionDefinition}]
  f(x: Int): Int = x + 16
  b = f(a)
  a = 1 + 2 + 4 + 8
\end{lstlisting}

In this way, $f$ is \textbf{defined completely and in exactly one place}.
This is how it is defined in \Soda.

Lambda functions can also be useful.
For example, the definition
$g = (\lambda x)(x + 32)$
would be defined as:
\begin{lstlisting}[label={lst:exampleFunctionDefinitionWithLambda}]
  g: Int -> Int = x -> x + 32
\end{lstlisting}

The right arrow (\srccode(->)) is used to define a domain mapping and also to define a lambda expression.


We can use a block to compute a value, for example:
\begin{lstlisting}[label={lst:exampleFunctionDefinitionWithBlock}]
  h0(x: Int, y: Int): Int = {
    a = x + y
    b = x - y
    a * b
  }
\end{lstlisting}
returns $(x + y)(x - y)$, which is $x^{2} - y^{2}$.

Although the order of the bindings does not matter, the result of the function has to be always at the end.
This is completely conventional.
For some people, the result at the beginning could be readable.
This can be somehow mimic by using a temporary constant stating that it contains the result.
Something like,
\begin{lstlisting}[label={lst:exampleFunctionDefinitionWithBlockAndResult}]
  h1(x: Int, y: Int): Int = {
    result = a * b
    a = x + y
    b = x - y
    result
  }
\end{lstlisting}

Standard arithmetic comparisons are also very useful.
For example,
\begin{lstlisting}[label={lst:exampleComparison}]
  geater_than(a: Int, b: Int): Boolean =
    a > b
\end{lstlisting}

This uses type \srccode{Boolean}, which can have values \sfalse and \strue.
The functions \sand, \sor, and \snot have the standard definitions.

Conditions can be also evaluated.
\begin{lstlisting}[label={lst:exampleMax}]
  max(a: Int, b: Int): Int =
    if a > b
    then a
    else b
\end{lstlisting}

We can group the constant and functions in classes.

\begin{lstlisting}[label={lst:exampleClassMax}]
  class MaxAndMin = {
    max(a: Int, b: Int): Int =
      if a > b
      then a
      else b

    min(a: Int, b: Int): Int =
      if a < b
      then a
      else b
  }
\end{lstlisting}

And we can have classes derived from \emph{abstract classes}.
The \emph{concrete classes} can produce instances.

\begin{lstlisting}[label={lst:exampleClassMaxWithIndex}]
  class ConcreteMaxAndMin() extends MaxAndMin
\end{lstlisting}

To create an instance we can just give the parameters, if needed.

\begin{lstlisting}[label={lst:exampleClassMaxAnInstance}]
  class MinMaxPair(min: Int, max: Int)

  class Example(index: Int) = {
    min_max(a: Int, b: Int): MinMaxPair =
      MinMaxPair(
        min := ConcreteMaxAndMin().min(a, b),
        max := ConcreteMaxAndMin().max(a, b)
      )
  }
\end{lstlisting}

Classes can be marked with \spackage reserved word and be put in \emph{packages}, which are collections of classes.
Classes can contain functions and other classes.

If a constant or function is to be implemented in another class, this can be denoted with \shas.

\begin{lstlisting}[label={lst:exampleAbstractFunction}]
  class Comparable = {
    has is_greater_than(x: Comparable): Boolean
  }
\end{lstlisting}


While the content in classes is public, the content in functions is private.
Because of that, to put classes outside functions would help to test them.

Basic Scala types are available (Int, Long, Boolean, Float, Double, String, \ldots).

Class definition can be also parameterized:
\begin{lstlisting}[label={lst:exampleParameterizedClass}]
  package example

  class Comparable = {
    has is_greater_than(x: Comparable): Boolean
  }

  class ComparableMax[T subtype Comparable]() = {
    max(a: T, b: T): T =
      if a.is_greater_than(b)
      then a
      else b
  }
\end{lstlisting}

Note that, considering packages, every constant, function, or class is defined in only one place, using the \sdef symbol.
It is not possible to add more information about the defined constant, function, or class in any other piece of code.
We can say that every constant, function, or class, is \emph{completely defined} in exactly one place.

In addition, every constant, function, or class is \emph{explicitly defined}.
This means, that all the defining components are explicitly available.

Let us see the following example:
\begin{lstlisting}[label={lst:exampleExplicitDefinition}]
  class MyClass (instance_parameter: Int) = {
    class_constant: Int = 1

    another_function (x: Int): Int = 2 * x

    class InnerClass() = {
      main_function (function_parameter: Int): Int =
        another_function(instance_parameter + class_constant + function_parameter)
    }
  }
\end{lstlisting}

The function \srccode{main\_function} is defined in exactly one place, inside \srccode{example.MyClass.InnerClass}, and to be fully defined it requires:
\begin{itemize}
    \item \srccode{another\_function}, defined in the same class
    \item \srccode{instance\_parameter}, a parameter given to create an instance
    \item \srccode{function\_parameter}, a parameter given when the function is invoked
\end{itemize}

The reserved words \ssubtype and \ssupertype put constraints in the parameterized datatype.

JDK classes can be imported, using the \simport command, and can be instantiated with \snew.

\begin{lstlisting}[label={lst:exampleJDKImport}]
  class TimeOfToday() {
    import java.util.Date

    get_time: Date = @new Date()
  }
\end{lstlisting}

Side effects are strongly discouraged, but they are possible through Java and Scala classes.

For example, a ``Hello world!'' would be:
\begin{lstlisting}[label={lst:exampleHelloWorld}]
  class Main() = {
    main (args: Array[String]) =
      println("Hello world!")
  }

  @main
\end{lstlisting}

The \smain annotation indicates the entry point of execution.

Another antipattern is overriding functionality.
It is strongly discouraged, but it is the only way of overriding \srccode{toString}, which is available to all objects.

This is:
\begin{lstlisting}[label={lst:exampleToString}]
  class PersonName (name: String) = {
    @override
    toString = name
  }
\end{lstlisting}

A tail recursion can be explicitly stated with a \stailrec annotation.

This language is meant to be concise, but using a reduced number of symbols.
However, there are a couple of useful synonyms to write a little less.

The symbol \sasterisk can be used instead of \sclass, regardless if it is abstract or concrete.
The symbol \splus can be used instead of \simport.

