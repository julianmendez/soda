\documentclass[12pt,a4paper]{article}

\usepackage[utf8]{inputenc}
\usepackage{hyperref}
\usepackage{amsmath}
\usepackage{listings}
\usepackage{xcolor}
\usepackage{xspace}


\lstdefinelanguage{Scopus}{
morekeywords={if, then, else, class, has, extends, with, this, false, true, not, and, or, package, import, new, is, in, to, that, equals, @override, @tailrec},
sensitive=true, % keywords are not case-sensitive
morecomment=[l]{//}, % l is for line comment
morecomment=[s]{/*}{*/}, % s is for start and end delimiter
morestring=[b]" % defines that strings are enclosed in double quotes
} %


\newcommand{\srccode}[1]{\texttt{{#1}}}
\newcommand{\reservedWord}[1]{{\color{blue}\srccode{#1}}\xspace}
\newcommand{\annotation}[1]{{\color{brown}\srccode{#1}}\xspace}

\newcommand{\sdef}{\srccode{=}}
\newcommand{\scolon}{\srccode{:}}

\newcommand{\sif}{\reservedWord{if}}
\newcommand{\sthen}{\reservedWord{then}}
\newcommand{\selse}{\reservedWord{else}}

\newcommand{\sclass}{\reservedWord{class}}
\newcommand{\shas}{\reservedWord{has}}
\newcommand{\sextends}{\reservedWord{extends}}
\newcommand{\swith}{\reservedWord{with}}
\newcommand{\sthis}{\reservedWord{this}}

\newcommand{\sfalse}{\reservedWord{false}}
\newcommand{\strue}{\reservedWord{true}}
\newcommand{\snot}{\reservedWord{not}}
\newcommand{\sand}{\reservedWord{and}}
\newcommand{\sor}{\reservedWord{or}}

\newcommand{\spackage}{\reservedWord{package}}
\newcommand{\simport}{\reservedWord{import}}
\newcommand{\snew}{\reservedWord{new}}

\newcommand{\stailrec}{\annotation{@tailrec}}
\newcommand{\soverride}{\annotation{@override}}

\newcommand{\sequalsSign}{\srccode{==}}

\newcommand{\sasterisk}{\srccode{*}}
\newcommand{\splus}{\srccode{+}}
\newcommand{\sminus}{\srccode{-}}

\newcommand{\sis}{\reservedWord{is}}
\newcommand{\scin}{\reservedWord{in}}
\newcommand{\sto}{\reservedWord{to}}
\newcommand{\sthat}{\reservedWord{that}}
\newcommand{\sequals}{\reservedWord{equals}}

\newcommand{\Scopus}{\textsc{Scopus}\xspace}
\newcommand{\lambdaSymbol}{{\tiny--\textgreater}}

\begin{document}

    \lstset{frame=tb,
    language=Scopus,
    aboveskip=3mm,
    belowskip=3mm,
    showstringspaces=false,
    columns=flexible,
    basicstyle={\small\ttfamily},
    numbers=none,
    numberstyle=\tiny\color{gray},
    keywordstyle=\color{blue},
    commentstyle=\color{green},
    stringstyle=\color{mauve},
    breaklines=true,
    breakatwhitespace=true,
    tabsize=3
    }


    \section{\Scopus}


    \section{Reserved words}

    The reserved words in \Scopus can be classified in the following categories:

    \begin{itemize}
        \item Functional
        \item Object Oriented
        \item Boolean
        \item Library
        \item Annotations
    \end{itemize}


    The reserved words considered in Scala are listed at this link:
    \url{https://www.scala-lang.org/files/archive/spec/2.13/01-lexical-syntax.html}

    \subsection{Functional}

    \subsubsection{= (definition symbol)}

    The definition symbol, written as an equals symbol, is used to defined functions and constants.

    For example,
    \begin{lstlisting}[label={lst:exampleDef}]
  a = 1
    \end{lstlisting}

    \subsubsection{: (type symbol)}

    The type symbol, written as a colon symbol, is used to assign a statically defined type to a function or constant.

    For example,
    \begin{lstlisting}[label={lst:exampleType01}]
  b: Int = 2
    \end{lstlisting}

    Another example is the following:
    \begin{lstlisting}[label={lst:exampleType02}]
  plus_one (x: Int): Int = x + 1
    \end{lstlisting}

    \subsubsection{\sif, \sthen, \selse}

    The \sif-\sthen-\selse construct is used to define a conditional result in a function.

    For example,
    \begin{lstlisting}[label={lst:exampleIfThenElse}]
  max (x: Int, y: Int) =
    if x > y
    then x
    else y
    \end{lstlisting}

    \subsubsection{\lambdaSymbol, $\rightarrow$ ($\backslash$u2192) (lambda symbol)}

    The lambda symbol, written as a right arrow, is used to bind a variable in a lambda expression.

    For example, this piece of code
    \begin{lstlisting}[label={lst:exampleLambda}]
  plus_one (sequence: Seq[Int]) =
    sequence.map(element -> element + 1)
    \end{lstlisting}
    returns a sequence of integers where each element is computed as the next integer from the input

    \subsection{Object Oriented}

    \subsubsection{\sclass, \shas}

    There are two types of classes: the abstract and the concrete.

    Abstract classes compare to interfaces in Java, or traits in Scala, and cannot be directly instantiated.

    Both types are declared using the \sclass reserved words.

    For example,
    \begin{lstlisting}[label={lst:exampleAbstractClass}]
  class Agent {
    has identifier: String
  }
    \end{lstlisting}
    defines an abstract agent that has some identifier.
    Agent is, however, an abstract class and cannot be instantiated.

    By contrast, concrete classes compare to final classes in Java and case classes in Scala.

    The following example shows a concrete class consisting only of a parameterized name.
    \begin{lstlisting}[label={lst:exampleConcreteClass}]
  class Person (name: String)
    \end{lstlisting}

    The difference between abstract classes and concrete classes is the parameters, possibly empty, that are required for concrete classes.

    \subsubsection{\sextends, \swith}

    It is possible to declare that a concrete class extends an abstract class using the \sextends reserved word.
    Concrete classes are final and cannot be extended.

    \begin{lstlisting}[label={lst:exampleExtends}]
  class Agent_Person (name: String) extends Agent {
    identifier = name
  }
    \end{lstlisting}

    Concrete classes can also extend multiple abstract classes.
    For example,
    \begin{lstlisting}[label={lst:exampleWith}]
  class Ranked_Individual {
    has rank: Int
  }

  class Ranked_Agent_Person (name: String, person_rank: Int) extends Agent with Ranked_Individual {
    identifier = name

    rank = person_rank
  }
    \end{lstlisting}

    \subsubsection{\sthis}

    In some cases, it could be necessary that an object can refer to itself.
    For that, the \sthis reserved word is used.

    For example,
    \begin{lstlisting}[label={lst:exampleWith}]
  class Element {
    has accept (v: Visitor): Boolean
  }

  class Visitor {
    has visit (x: Element): Boolean
  }

  class Item (identifier: Int) extends Element {
    accept (v: Visitor) = v.visit (this)
  }
    \end{lstlisting}

    \subsection{Boolean}

    \subsubsection{\sfalse, \strue}

    As in other programming languages, the \sfalse and \strue reserved words are used for the Boolean values.

    \begin{lstlisting}[label={lst:exampleFalseTrue}]

  my_not (x: Boolean) =
    if x
    then false
    else true

  my_and (x: Boolean, y: Boolean) =
    if x
    then
      if y
      then true
      else false
    else false

  my_or (x: Boolean, y: Boolean) =
    if x
    then true
    else
      if y
      then true
      else false
    \end{lstlisting}

    \subsubsection{\snot, \sand, \sor}

    The reserved words \snot, \sand, \sor are introduced for the sake of readability.

    \begin{lstlisting}[label={lst:exampleNotAndOr}]

  my_xor (x: Boolean, y: Boolean) = (x or y) and not (x and y)

    \end{lstlisting}

    \subsection{Library}

    \subsubsection{\spackage}

    The \spackage declares the package the content should be included.
    All classes then belong to that package.

    \begin{lstlisting}[label={lst:examplePackage}]
  package org.example.scopus
    \end{lstlisting}

    \subsubsection{\simport}

    The \simport helps importing classes from other libraries and frameworks.

    \begin{lstlisting}[label={lst:examplePackage}]
  import java.util.Date

  import scala.annotation.tailrec
    \end{lstlisting}

    \subsubsection{\snew}

    Although every concrete class can be instantiated directly using the parameters, some libraries and frameworks may need a \snew reserved word.
    For example,

    \begin{lstlisting}[label={lst:examplePackage}]
  import java.util.Date

  now = new Date()
    \end{lstlisting}

    \subsection{Annotations}

    Annotations are not part of the language itself, but they are necessary to translate it into Scala.

    \subsubsection{\stailrec}

    The \stailrec annotation is used to ensure that a recursive function uses tail recursion.
    If a tail recursive function is not properly converted to a loop, it could produce a stack overflow after few iterations.

    For technical reasons, the \stailrec annotation needs to be imported.

    For example, the following example shows a function \srccode{sum} that returns the sum of the first $n$ natural numbers, or 0 if $n \leq 0$.

    \begin{lstlisting}[label={lst:exampleTailrec}]
  import scala.annotation.tailrec

  sum (n: Int) = sum_rec (n, 0)

  @tailrec
  sum_rec (n: Int, accum: Int): Int =
    if n < 0
    then accum
    else sum_rec (n - 1, n + accum)

    \end{lstlisting}

    \subsubsection{\soverride}

    The \soverride annotation is used to ensure that a function is effectively overriding another function.
    When programming, a function name could be mispelled or could be declared in another abstract class.
    This annotation prevents this problem.

    \begin{lstlisting}[label={lst:exampleOverride}]
  class Person_Name (name: String) {
    @override
    toString = name
  }
    \end{lstlisting}


    \section{Synonyms}

    One of the interesting features of \Scopus is the use of synonyms.
    They can be used to shape the code for a particular domain.

    The synonyms found only at the beginning of a line are:

    \begin{center}
        \begin{tabular}{|ll|}
            \hline
            \textbf{synonym} & \textbf{reserved word} \\
            \hline
            \sasterisk       & \sclass                \\
            \hline
            \sminus          & \shas                  \\
            \hline
            \splus           & \simport               \\
            \hline
        \end{tabular}
    \end{center}

    The synonyms that can be at any part of a line are:

    \begin{center}
        \begin{tabular}{|ll|}
            \hline
            \textbf{synonym} & \textbf{reserved word} \\
            \hline
            \sis             & \sdef                  \\
            \hline
            \scin            & \scolon                \\
            $\in$            & \scolon                \\
            \hline
            \sto             & \lambdaSymbol          \\
            \sthat           & \lambdaSymbol          \\
            $\to$            & \lambdaSymbol          \\
            \hline
            \sequals         & \sequalsSign           \\
            \hline
        \end{tabular}
    \end{center}

\end{document}


