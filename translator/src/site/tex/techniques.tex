\section{Techniques}

This section is intended to describe techniques to produce \Soda source code.
This could be considered simple design patterns that use the features of \Soda.

One of the techniques is to have a kind of \textit{weak scope}.

Let us assume we have the following snippet:
\begin{lstlisting}[label={lst:techniquesWeakScope0}]
  class SomeClass = {

    relevant_function(x: Input) =
      let
        auxiliary_function_1 = ...
        auxiliary_function_2 = ...
        auxiliary_function_3 = ...
        ...
      in result


    unrelated_function_1 = ...

    unrelated_function_2 = ...
  }
\end{lstlisting}

As we can see, the auxiliary functions are \textbf{private}, and cannot be tested outside \srccode{relevant\_function}.
An alternative, would be to put the auxiliary functions in the class, but outside \srccode{relevant\_function}.

\begin{lstlisting}[label={lst:techniquesWeakScope1}]
  class SomeClass = {

    relevant_function(x: Input) =
      let
        ...
      in result

    auxiliary_function_1 = ...

    auxiliary_function_2 = ...

    auxiliary_function_3 = ...

    unrelated_function_1 = ...

    unrelated_function_2 = ...
  }
\end{lstlisting}

This works, but the auxiliary functions are mixed together with the unrelated functions.
This is when the weak scope pattern can be useful.

\begin{lstlisting}[label={lst:techniquesWeakScope2}]
  class SomeClass = {

    relevant_function(x: Input) =
      let
        ...
      in result

    class _AuxRF() = {

      auxiliary_function_1 = ...

      auxiliary_function_2 = ...

      auxiliary_function_3 = ...
    }

    unrelated_function_1 = ...

    unrelated_function_2 = ...
  }
\end{lstlisting}

In this case, \srccode{relevant\_function} can use the auxiliary functions in class \srccode{\_AuxRF}.
These functions can be tested from outside, but it is clear that its content is not meant to be exported.

\subsection{Common patterns when writing in \Soda}
\begin{itemize}
    \item use of flatMap / ... / flatMap / map
    \item use abstract classes to contain constants
\end{itemize}

\subsection{How to translate Scala source code into \Soda}

\begin{itemize}
    \item remove any use of 'null'
    \item make the source code completely functional (no 'var', no 'for', no 'while')
    \item optimize source code and clean up imports
    \item make all classes either 'case class' or 'trait', and make any class renaming if necessary
    \item fix syntax for comments and Scaladoc
    \item remove all braces possible
    \item replace 'enum' and create traits with constants
    \item move imports into the classes that use them
    \item take inner classes outside their parent classes, and add an underscore prefix if they are meant to be protected
    \item take inner functions outside and make them protected adding an underscore prefix
    \item rename private functions and add an underscore prefix
    \item replace the 'object' constructs, for example, by traits with constants and builder case classes
    \item replace 'try' and 'catch' by Scala class Try
    \item replace Scala Try by Option
    \item replace 'for' comprehension expressions by flatMap/map combinations
    \item replace use of underscore '\_'
    \item replace 'case' and 'match' reserved words, if it is a type casting, it could be done:
    if (obj.isInstanceOf[SpecificType]) {
        map = obj.asInstanceOf[SpecificType]
    }
    \item copy Scala files as Soda files
    \item remove 'def', remove 'val' and 'final val'
    \item replace 'case class' by '*', replace 'trait' by '*', replace 'import' by '+', add '=' in class declarations,
          replace 'new' by '@new', replace '=>' by '->', replace '=' in parameters by ':=',
          replace '\&\&' by 'and', replace '||' by 'or', replace '!' by 'not'
    \item adapt 'if-then-else' structures
    \item add '@main' in 'Main' class
    \item replace braces by 'let-in' structures
\end{itemize}

