\section{Techniques}

This section is intended to describe techniques to produce \Soda source code.
This could be considered simple design patterns that use the features of \Soda.

One of the techniques is to have a kind of \textit{weak scope}.

Let us assume we have the following snippet:
\begin{lstlisting}[label={lst:techniquesWeakScope0}]
  class SomeClass = {

    relevant_function(x: Input) =
      let
        auxiliary_function_1 = ...
        auxiliary_function_2 = ...
        auxiliary_function_3 = ...
        ...
      in result


    unrelated_function_1 = ...

    unrelated_function_2 = ...
  }
\end{lstlisting}

As we can see, the auxiliary functions are \textbf{private}, and cannot be tested outside \srccode{relevant\_function}.
An alternative, would be to put the auxiliary functions in the class, but outside \srccode{relevant\_function}.

\begin{lstlisting}[label={lst:techniquesWeakScope1}]
  class SomeClass = {

    relevant_function(x: Input) =
      let
        ...
      in result

    auxiliary_function_1 = ...

    auxiliary_function_2 = ...

    auxiliary_function_3 = ...

    unrelated_function_1 = ...

    unrelated_function_2 = ...
  }
\end{lstlisting}

This works, but the auxiliary functions are mixed together with the unrelated functions.
This is when the weak scope pattern can be useful.

\begin{lstlisting}[label={lst:techniquesWeakScope2}]
  class SomeClass = {

    relevant_function(x: Input) =
      let
        ...
      in result

    class _AuxRF() = {

      auxiliary_function_1 = ...

      auxiliary_function_2 = ...

      auxiliary_function_3 = ...
    }

    unrelated_function_1 = ...

    unrelated_function_2 = ...
  }
\end{lstlisting}

In this case, \srccode{relevant\_function} can use the auxiliary functions in class \srccode{\_AuxRF}.
These functions can be tested from outside, but it is clear that its content is not meant to be exported.

