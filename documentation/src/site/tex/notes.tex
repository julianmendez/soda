\section{Why and why not}


\subsection{Mutability}
Mutable structures are extremely error prone, since a variable can change its value without any apparent change.

Example:

/* a = ?, b = ?, c = ? */
a = C(5)

/* a = C(5), b = ?, c = ? */
b = compute(a)

/* a = C(6), b = C(a), c = ? */
c = compute(b)
/* a = C(6), b = C(7), c = C(7) */


compute(a)
a.increment(1)
return a


expected
/* a = ?, b = ?, c = ? */
a = C(5)

/* a = C(5), b = ?, c = ? */
b = compute(a)

/* a = C(5), b = C(6), c = ? */
c = compute(b)
/* a = C(5), b = C(6), c = C(7) */



\subsection{Tuples}

The use of tuple can hide an error in design.
Unlike sequences, tuples may accept different types in their components.

For example,

("Jon", true, true, false)

is far less legible than

Individual(name := "Jon", registered := true, evaluated := true, passed := true)


