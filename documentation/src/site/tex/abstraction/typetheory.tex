\section{Type Theory}

\Soda has pure functional notation.
Correctness of functions can be proven using an external tool like Coq or Isabelle.
In addition, functions can include some sort of \textit{assertion} that checks in runtime that the function has properly computed the input.
For example, a \srccode{sort} function can be later called by a \srccode{is\_sorted} function, to verify that the sorting function correctly sort a sequence.
The return of this check could be an \srccode{OptionSD}, so that this needs to be handled in case of an error.

\Soda can be used to ``prove'' theorems by using its type system.

There is some relation between Proof Theory in Philosophy, Category Theory in Mathematics, and Type Theory in Computer Science.

It is possible to define type subsumption in \Soda, by using \sextends.

Then, we can say that $A \ \sextends\ B$, is like saying $A \subseteq B$, or $A \to B$, or $A \sqsubseteq B$.
Analogously, $A \ \sextends\ B \ \swith\ C$, is like saying $A \subseteq B \cap C $, or $A \to B \land C$, or $A \sqsubseteq B \sqcap C$.

We have that $\srccode{Any}$, is like saying $U$, or $True$, or $\top$.
Likewise with $\srccode{Nothing}$, is like saying $\emptyset$, or $False$, or $\bot$.

At the moment, it is part of the future work how to express more complex constraints in \Soda.

Some of the mapping are still open question for \Soda.
These are some questions for the case of description logics.

\begin{itemize}
    \item How can we say $A \sqsubseteq \exists r. B $
    \item How to define a \textit{TBox}?
    \item What does an \textit{ABox} mean?
    \item Can we detect unsuitable types by checking its emptiness?
    \item What is the meaning of $A \sqsubseteq \lnot B$?
    \item What is the meaning of $A \sqsubseteq \exists r. A$?
    \item Is it possible to relate the roles with the class methods?
    \item How do abstract and concrete classes play?
    \item How could a \textit{TBox} be translated into Scala code?
\end{itemize}

\begin{itemize}
    \item Correct code generation in Scala:
    \item \url{https://logika.sireum.org/}
    \item \url{https://github.com/JBakouny/Scallina}
\end{itemize}

