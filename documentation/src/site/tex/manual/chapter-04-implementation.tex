\chapter{Implementation}

\Soda (Symbolic Objective Descriptive Analysis) is a human-centered formalism to describe, formalize, and prototype ethical problems and constraints.
It could be seen as a specification language or as a functional programming language.

The reserved words in \Soda can be classified in the following categories:

\begin{itemize}
    \item Functional
    \item Object Oriented
    \item Boolean
    \item Library
    \item Annotations
\end{itemize}


%The reserved words considered in Scala are listed at this link:
%\url{https://www.scala-lang.org/files/archive/spec/2.13/01-lexical-syntax.html}

This is a tentative logo:

\Sodalogo


\section{Functional}

\subsubsection{= (definition symbol)}

The definition symbol, written as an equals symbol, is used to defined functions and constants.

For example,
\begin{lstlisting}[label={lst:exampleDef}]
  a = 1
\end{lstlisting}

\subsubsection{: (type symbol)}

The type symbol, written as a colon symbol, is used to assign a statically defined type to a function or constant.

For example,
\begin{lstlisting}[label={lst:exampleType01}]
  b : Int = 2
\end{lstlisting}

Another example is the following:
\begin{lstlisting}[label={lst:exampleType02}]
  plus_one (x : Int) : Int = x + 1
\end{lstlisting}

\subsubsection{\sodaif, \sodathen, \sodaelse}

The \sodaif-\sodathen-\sodaelse construct is used to define a conditional result in a function.

For example,
\begin{lstlisting}[label={lst:exampleIfThenElse}]
  max (x : Int) (y : Int) : Int =
    if x > y
    then x
    else y
\end{lstlisting}

\subsubsection{\sodaarrow (lambda symbol)}

The lambda symbol, written as a right arrow, is used to bind a variable in a lambda expression.

For example, this piece of code
\begin{lstlisting}[label={lst:exampleLambda}]
  plus_one (sequence : Seq [Int] ) : Seq [Int] =
    sequence.map (element -> element + 1)
\end{lstlisting}
returns a sequence of integers where each element is computed as the next integer from the input


\section{Object Oriented}

\subsubsection{\sodaclass, \sodaabstract}

As mentioned, there are two types of classes: the abstract and the concrete.

Abstract classes compare to interfaces in Java, or traits in Scala, and cannot be directly instantiated.

Both types are declared using the \sodaclass reserved words.

For example,
\begin{lstlisting}[label={lst:exampleAbstractClass}]
class Agent

  abstract
    identifier : String

end
\end{lstlisting}
defines an abstract agent that has some identifier.
Agent is an abstract class and cannot be instantiated.

By contrast, concrete classes compare to final classes in Java and case classes in Scala.

The following example shows a class consisting only of a parameterized name.
\begin{lstlisting}[label={lst:exampleConcreteClass}]
class Person

    abstract
      name : String

end
\end{lstlisting}

The difference between abstract classes and concrete classes is the parameters, possibly empty, that are required for concrete classes.

\subsubsection{\sodaextends}

It is possible to declare that a concrete class extends an abstract class using the \sodaextends reserved word.
Concrete classes are final and cannot be extended.

\begin{lstlisting}[label={lst:exampleExtends}]
class AgentPerson
  extends
    Agent

  abstract
    name: String

  identifier = name

end
\end{lstlisting}

Concrete classes can also extend multiple abstract classes.
For example,
\begin{lstlisting}[label={lst:exampleWith}]
class RankedIndividual

  abstract
    rank: Int

end
\end{lstlisting}

\subsubsection{\sodathis}

In some cases, it could be necessary that an object can refer to itself.
For that, the \sodathis reserved word is used.

For example,
\begin{lstlisting}[label={lst:exampleExtendsElement}]
class Element

  abstract
    accept (v : Visitor) : Boolean

end

class Visitor

  abstract
    visit (x : Element) : Boolean

end

class Item
  extends
    Element

  abstract
    identifier : Int

  accept (v : Visitor) = v.visit (this)

end
\end{lstlisting}

\subsubsection{\sodasubtype, \sodasupertype}

The bounds of a parametric type can be determined with \sodasubtype, for the upper bound, and \sodasupertype, for the lower bound.


\section{Boolean}

\subsubsection{\sodafalse, \sodatrue}

As in other programming languages, the \sodafalse and \sodatrue reserved words are used for the Boolean values.

\begin{lstlisting}[label={lst:exampleFalseTrue}]
  my_not (x : Boolean) : Boolean =
    if x
    then false
    else true

  my_and (x : Boolean) (y : Boolean) : Boolean =
    if x
    then y
    else false

  my_or (x : Boolean) (y : Boolean) : Boolean =
    if x
    then true
    else y
\end{lstlisting}

\subsubsection{\sodanot, \sodaand, \sodaor}

The reserved words \sodanot, \sodaand, \sodaor are introduced for the sake of readability.

\begin{lstlisting}[label={lst:exampleNotAndOr}]

  my_xor (x : Boolean) (y : Boolean) : Boolean =
    (x or y) and not (x and y)

\end{lstlisting}


\section{Library}

\subsubsection{\sodapackage}

The reserved word \sodapackage declares the package where the content should be included.
All classes then belong to that package.

\begin{lstlisting}[label={lst:examplePackage}]
package org.example.soda
\end{lstlisting}

\subsubsection{\sodaimport}

The reserved word \sodaimport helps importing classes from other libraries and frameworks.
It is recommended to put the \sodaimport inside the class that is using it.
This differs from common practice in Java, where all the import statements are at the beginning of the file, outside the class.
It is important to notice that the imports can always be moved outwards, i.e. from inside the class to outside.
Including the imports in the class and not in more general place helps removing a class with its own imports.
If these imports are in general place, e.g. visible from the whole package, it becomes harder to maintain when classes are added or removed.
To have a view of the imports needed by a package, it would be necessary to use a tool, like an IDE.


\begin{lstlisting}[label={lst:exampleImport}]
  import
    java.util.Date
\end{lstlisting}


\section{Annotations}

Annotations are not part of the language itself, but they are necessary to translate it into Scala.

\subsubsection{\sodanew}

Although every concrete class can be instantiated directly using the parameters, some JVM libraries and frameworks may need a \sodanew reserved word.
This is only necessary when the code is translated into Scala 2.
For translations to Scala 3, the \sodanew annotation can be omitted.
For example,

\begin{lstlisting}[label={lst:exampleImportDate}]
  import
    java.util.Date

  now = @new Date ()
\end{lstlisting}

\subsubsection{\sodaoverride}

The \sodaoverride annotation may be necessary to override some JVM functions, like \srccode{toString()}.
However, this could be misused to override functions and values of abstract classes.
This could make that a concrete class does not behave as the abstract class that implements.
For this reason, although \sodaoverride annotation could be used to ensure that a function is effectively overriding another function, it should only be used to override JVM functions, and not \Soda functions.

\begin{lstlisting}[label={lst:exampleOverride}]
class PersonName (name: String) =

    @override
    toString = name

end
\end{lstlisting}

\subsubsection{\sodatailrec}

The \sodatailrec annotation is used to ensure that a recursive function uses tail recursion.
If a tail recursive function is not properly converted to a loop, it could produce a stack overflow after few iterations.

\begin{lstlisting}[label={lst:exampleTailrecInside}]
  @tailrec
  _tailrec (n : Int) (accum : Int) : Int =
    if n < 0
    then accum
    else _tailrec (n - 1) (n + accum)

  sum (n : Int) : Int =
    _tailrec (n, 0)
\end{lstlisting}

\subsubsection{Main}

The Main class is used to indicate the entry point.
This is kept to have functions that can be executed from the console.

\begin{lstlisting}[label={lst:exampleMain}]
class Main

  main (args : Array[String] ) : Unit =
    println ("Hello world!")

end
\end{lstlisting}

This \srccode{main} method can be accessed by invoking a statically accessible \srccode{main} method in class \srccode{EntryPoint} located in the same package.

