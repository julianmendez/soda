\chapter{Usage and Methodology}

\section{Techniques}

This section is intended to describe techniques to produce \Soda source code.
This could be considered simple design patterns that use the features of \Soda.

One of the techniques is to have a kind of \textit{weak scope}.

Let us assume we have the following snippet:
\begin{lstlisting}[label={lst:techniquesWeakScope0}]
  class SomeClass = {

    relevant_function(x: Input) =
      let
        auxiliary_function_1 = ...
        auxiliary_function_2 = ...
        auxiliary_function_3 = ...
        ...
      in result


    unrelated_function_1 = ...

    unrelated_function_2 = ...
  }
\end{lstlisting}

As we can see, the auxiliary functions are \textbf{private}, and cannot be tested outside \srccode{relevant\_function}.
An alternative, would be to put the auxiliary functions in the class, but outside \srccode{relevant\_function}.

\begin{lstlisting}[label={lst:techniquesWeakScope1}]
  class SomeClass = {

    relevant_function(x: Input) =
      let
        ...
      in result

    auxiliary_function_1 = ...

    auxiliary_function_2 = ...

    auxiliary_function_3 = ...

    unrelated_function_1 = ...

    unrelated_function_2 = ...
  }
\end{lstlisting}

This works, but the auxiliary functions are mixed together with the unrelated functions.
This is when the weak scope pattern can be useful.

\begin{lstlisting}[label={lst:techniquesWeakScope2}]
  class SomeClass = {

    relevant_function(x: Input) =
      let
        ...
      in result

    class _AuxRF() = {

      auxiliary_function_1 = ...

      auxiliary_function_2 = ...

      auxiliary_function_3 = ...
    }

    unrelated_function_1 = ...

    unrelated_function_2 = ...
  }
\end{lstlisting}

In this case, \srccode{relevant\_function} can use the auxiliary functions in class \srccode{\_AuxRF}.
These functions can be tested from outside, but it is clear that its content is not meant to be exported.

\subsection{Common patterns when writing in \Soda}
\begin{itemize}
    \item use of flatMap / ... / flatMap / map
    \item use abstract classes to contain constants
\end{itemize}

\subsection{How to translate Scala source code into \Soda}

\begin{itemize}
    \item remove any use of 'null'
    \item make the source code completely functional (no 'var', no 'for', no 'while')
    \item optimize source code and clean up imports
    \item make all classes either 'case class' or 'trait', and make any class renaming if necessary
    \item fix syntax for comments and Scaladoc
    \item remove all braces possible
    \item replace 'enum' and create traits with constants
    \item move imports into the classes that use them
    \item take inner classes outside their parent classes, and add an underscore prefix if they are meant to be protected
    \item take inner functions outside and make them protected adding an underscore prefix
    \item rename private functions and add an underscore prefix
    \item replace the 'object' constructs, for example, by traits with constants and builder case classes
    \item replace 'try' and 'catch' by Scala class Try
    \item replace Scala Try by Option
    \item replace 'for' comprehension expressions by flatMap/map combinations
    \item replace use of underscore '\_'
    \item replace 'case' and 'match' reserved words, if it is a type casting, it could be done:
    if (obj.isInstanceOf[SpecificType]) {
        map = obj.asInstanceOf[SpecificType]
    }
    \item copy Scala files as Soda files
    \item remove 'def', remove 'val' and 'final val'
    \item replace 'case class' by '*', replace 'trait' by '*', replace 'import' by '+', add '=' in class declarations,
    replace 'new' by '@new', replace '=>' by '->', replace '=' in parameters by ':=',
    replace '\&\&' by 'and', replace '||' by 'or', replace '!' by 'not'
    \item adapt 'if-then-else' structures
    \item add '@main' in 'Main' class
    \item replace braces by 'let-in' structures
\end{itemize}

\subsection{How to rewrite \Soda code}

One the powerful things in \Soda is the capability of rewriting code.
Remember that its purpose is to be a easy to understand to humans as possible.

The goal can be summarized as:

\begin{itemize}
    \item use small functions and constants with descriptive and not too long names
    (the more specific a publicly accessible function is, the longer its name)
    \item use small classes with descriptive and not too long names
    \item try to make every function do one thing
    \item try to make every class model one thing
    \item prefer abstract classes unless an instance is required
    \item avoid code duplication
    \item avoid parametric types, unless is significantly reduces code duplication
    \item avoid complex structures, like many successive nested if-then-else structures
\end{itemize}

\subsubsection{Function extraction}

If a function is not specific of another function, it can be extracted

For example,

\begin{lstlisting}[label={lst:rewritingFunctionExtractionBefore}]
  class SomeClass = {

    relevant_function(x: Input) =
      let
        a = f(x)
        b = some_constant
        result = a + b
      in result
  }
\end{lstlisting}

since function b does not depend on $x$ or $a$, it can be extracted

\begin{lstlisting}[label={lst:rewritingFunctionExtractionAfter0}]
  class SomeClass = {

    b = some_constant

    relevant_function(x: Input) =
      let
        a = f(x)
        result = a + b
      in result
  }
\end{lstlisting}

and it can be renamed to be more descriptive

\begin{lstlisting}[label={lst:rewritingFunctionExtractionAfter1}]
  class SomeClass = {

    specific_constant_for_this_class = some_constant

    relevant_function(x: Input) =
      let
        a = f(x)
        result = a + specific_constant_for_this_class
      in result
  }
\end{lstlisting}

\subsubsection{Function replacement}

If a function is too complex, we can replace a piece by another function.

For example,

\begin{lstlisting}[label={lst:rewritingFunctionReplacementBefore}]
  class SomeClass = {

    relevant_function(x: Input) =
      let
        a: TypeOfF = f(x)
        b =
          if some_condition(a)
          then some_value(x)
          else some_other_value(x)
        result = a + b
      in result
  }
\end{lstlisting}

since function $b$ depends on $x$ and $a$, it can be replaced by $new\_function$

\begin{lstlisting}[label={lst:rewritingFunctionReplacementAfter}]
  class SomeClass = {

    new_function(x: Input, a: TypeOfF) =
      if some_condition(a)
      then some_value(x)
      else some_other_value(x)

    relevant_function(x: Input) =
      let
        a: TypeOfF = f(x)
        b = new_function(x, a)
        result = a + b
      in result
  }
\end{lstlisting}

\subsubsection{Class split}

A class can be split to make two classes more specific

Let us assume that there two groups of functions $f_{i}$ and $g_{i}$, where the $f_{i}$ functions do not depend on the $g_{i}$ and the $g_{i}$ functions do not depend on the $f_{i}$.
Such a class could look like:

\begin{lstlisting}[label={lst:rewritingClassSplitBefore}]
  class SomeClass = {

    f_0 = ...
    f_1 = ...
    ...
    f_n = ...

    g_0 = ...
    g_1 = ...
    ...
    g_m = ...
  }
\end{lstlisting}

We can then split the class as:

\begin{lstlisting}[label={lst:rewritingClassSplitAfter}]
  class SomeClassWithF = {

    f_0 = ...
    f_1 = ...
    ...
    f_n = ...
  }

  class SomeClassWithG = {

    g_0 = ...
    g_1 = ...
    ...
    g_m = ...
  }

  class SomeClass
    extends SomeClassWithF
    with SomeClassWithG
\end{lstlisting}

\subsubsection{Class generalization}

Every concrete class can be split between its abstract and its concrete part.

\begin{lstlisting}[label={lst:rewritingClassGeneralizationBefore}]
  class SomeClass(v_0: Type_0, ..., v_n: Type_n) = {
    f_0 = ...
    ...
    f_m = ...
  }
\end{lstlisting}

can be split as

\begin{lstlisting}[label={lst:rewritingClassGeneralizationAfter}]
  class AbstractPartOfSomeClass = {
    has v_0: Type_0
    ...
    has v_n: Type_n

    f_0 = ...
    f_1 = ...
    ...
    f_n = ...
  }

  class SomeClass(v_0: Type_0, ..., v_n: Type_n)
    extends AbstractPartOfSomeClass
\end{lstlisting}


\section{Why and why not}

\subsection{Mutability}
Mutable structures are extremely error-prone, since a variable can change its value without any apparent change.

Example:

/* a = ?, b = ?, c = ? */
a = C(5)

/* a = C(5), b = ?, c = ? */
b = compute(a)

/* a = C(6), b = C(a), c = ? */
c = compute(b)
/* a = C(6), b = C(7), c = C(7) */


compute(a)
a.increment(1)
return a


expected
/* a = ?, b = ?, c = ? */
a = C(5)

/* a = C(5), b = ?, c = ? */
b = compute(a)

/* a = C(5), b = C(6), c = ? */
c = compute(b)
/* a = C(5), b = C(6), c = C(7) */

\subsection{Tuples}

The use of tuple can sometimes hide an error in design.
Unlike sequences, tuples may accept different types in their components.

For example,

\begin{lstlisting}[label={lst:exampleJonTuple}]
  ("Jon", true, true, false)
\end{lstlisting}

is far less legible than

\begin{lstlisting}[label={lst:exampleJonIndividual}]
  Individual(name := "Jon", registered := true, evaluated := true, passed := true)
\end{lstlisting}

\subsection{Significant whitespace}

In Python, horizontal tabulations can make a big difference.

This is the intended snippet:

\begin{lstlisting}[label={lst:snippetInPython1}]
  if x > 0:
    positive = True
    y = x
  else:
    positive = False
    y = 0
\end{lstlisting}


The following function has a bug, since it always assigns 0 to variable $y$.

\begin{lstlisting}[label={lst:snippetInPython2}]
  if x > 0:
    positive = True
    y = x
  else:
    positive = False
  y = 0
\end{lstlisting}

This kind of bug is for a 6-line snippet.
This can propagate when thousands of lines are written.

In Scala 3, it is possible to define classes using a similar criterion.

If we used this type of notation in \Soda, this could produce a similar type of error:

Original expected declaration

\begin{lstlisting}[label={lst:classesWithBraces}]
class A = {
  f_a = 0
}

class B = {
  f_b = 1
}
\end{lstlisting}

With this notation
\begin{lstlisting}[label={lst:classesWithColons}]
class A:
  f_a = 0

class B:
  f_b = 1
\end{lstlisting}

With a bug:
\begin{lstlisting}[label={lst:classesWithColonsAndMistake}]
class A:
  f_a = 0

  class B:
    f_b = 1
\end{lstlisting}

In the last example, class B is part of class A.
The compiler will accept it, and a human can accept it.

